\documentclass[compress, aspectratio=169]{beamer}

%presentation layout

\mode<presentation>
{
  \usetheme{Berlin}
  % \usecolortheme{dove}
  \setbeamercolor{structure}{bg=black,fg=white}
  \setbeamercolor{normal text}{bg=black,fg=white}
  \setbeamercolor{titlepage}{bg=black,fg=white}
  \setbeamercolor{titlelike}{bg=black,fg=white}
  \setbeamercolor{palette primary}{bg=black}
  \setbeamercolor{palette secondary}{bg=black, fg=gray}
  \setbeamercolor{palette tertiary}{bg=black, fg=gray}
  \setbeamercolor{palette quarternary}{bg=black}
  \setbeamercovered{transparent}
  \useinnertheme{rectangles}
  %\usefonttheme{serif}
}

\setbeamertemplate{navigation symbols}{}

%loading packages
\usepackage[ngerman]{babel}
\usepackage[T1]{fontenc}
\usepackage[utf8]{inputenc}
\usepackage{graphicx}
\usepackage{amsmath}
\usepackage{framed}

% vorgeplaenkel
\title[StAPf-Bericht]{StAPF-Bericht}

\author{Ständiger Ausschuss aller Physikfachschaften}

\institute[Zusammenkunft aller Physikfachschaften]

\date{28. Oktober 2017}

\subject{ZäPFchen-Einführung}

\begin{document}

\begin{frame}[plain]{}
  \titlepage
\end{frame}

\section{Zusammensetzung}

\begin{frame}{Aktuelle Zusammensetzung}
  \begin{itemize}
  \item \emph{Peter Steinmüller} (KIT Karlsruhe)
  \item \emph{Katharina Meixner} (Uni Frankfurt)
  \item Niklas Donocik (TU Braunschweig)
  \item Jennifer Hartfiel (FU Berlin)
  \item Oliver Irtenkauf (Uni Konstanz)
  \end{itemize}
\end{frame}

\section{Was bisher geschah...}

\begin{frame}{Was bisher geschah...}
  \begin{itemize}
  \item Resolutionen verschickt und veröffentlicht
    \begin{itemize}
    \item Kritik am Besetzungsverhalten der HRK in den Akkreditierungsrat
    \item Resolution gegen Studiengebühren
    \item Resolution zur Schaffung permanenter Stellen im wissenschaftlichen Mittelbau
    \item Resolution zur studentischen Beteiligung bei Bauvorhaben
    \item Resolution zur Exzellenz-Strategie
    \item Öffentlicher Brief zum Thema VG-WORT
    \end{itemize}
  \end{itemize}
\end{frame}

\begin{frame}
  \begin{itemize}
    \item ZaPF-Bericht verschickt und veröffentlicht
    \item 7 Sitzungen seit der ZaPF in Berlin
    \item Klausurtagung vom 07.07. bis 09.07. in Siegen zur Nachbereitung von Berlin
    \item Klausurtagung vom 29.09. bis 01.10. in Berlin zur Vorbereitung von Siegen
    \end{itemize}
    \vspace{5mm}
    \begin{center}
      \Large DANKE an alle, die uns dabei unterstützt haben!
    \end{center}
\end{frame}

\section{Rückmeldungen}

\begin{frame}{Resolution zum Akkreditierungsrat}
  \begin{itemize}
  \item Es wurde eine Pressemitteilung veröffentlicht.
  \item Auf dem letzten Poolvernetzungstreffen wurden erneut studentische Vertreter gewählt und dem Akkreditierungsrat vorgeschlagen.
  \item Es gibt noch keine Reaktion.
  \end{itemize}
\end{frame}

\section{VG Wort}

\begin{frame}{Aktueller Stand zu VG Wort}
  \begin{itemize}
    \item Das Moratorium wurde bis 01. März 2018 verlängert.
    \item Ab dann gibt es ein neues Urheberrecht, in dem eine Pauschalabrechnung eingebaut ist
  \end{itemize}
\end{frame}

\section{Akkreditierung}

%unvollständig
\begin{frame}{Akkreditierungspool}
  \begin{itemize}
    \item[$\rightarrow$] Auslaufende Mandate:
      \begin{itemize}
      \item Markus Gleich
      \item Margret Heinze
      \item Björn Guth
      \item Thomas Kirchner
      \item Katharina Meixner
      \item Jannis Andrija Schnitzer
      \end{itemize}
    \item PVT
      \begin{itemize}
      \item Letztes Treffen 06./07.08.2016 in Kiel\\
        Es waren zwei ZaPFika dort.
      \item Nächstes Treffen steht noch nicht fest.
      \end{itemize}
    \end{itemize}
\end{frame}

\begin{frame}
  \begin{framed}
    \begin{center}
      {\Huge \textbf{Wichtig}}\\
      \vspace{0.5cm}
      {\Large Aktuelle Anmeldeformulare für den Pool ausfüllen und in digitaler Form an die Verwaltung senden}
      \end{center}
      \end{framed}
\end{frame}

% \section{MeTaFa}

% \begin{frame}{MeTaFa}
%   \begin{itemize}
%   \item Treffen in Dresden (22. bis 24. September 2017 )
%     \begin{itemize}
%       \item Die ZaPF war dieses Mal nicht vertreten
%     \end{itemize}
%   \end{itemize}
% \end{frame}

\section{Kommende ZaPFen}
\begin{frame}{Kommende ZaPFen}
  \begin{itemize}
    \item Sommersemester 2018 in Heidelberg
    \item Wintersemester 2018 in Würzburg
    \item zu Vergeben:
      \begin{itemize}
      \item Sommersemester 2019
      \item ...
      \end{itemize}
    \item bisherige Bewerbungen:
      \begin{itemize}
      \item Hier könnte deine Uni stehen!
      \end{itemize}
    \end{itemize}
\end{frame}

\begin{frame}[plain]
  \begin{center}
    \Huge Habt ihr Fragen an uns?
    \end{center}
\end{frame}

\section{TOPF}
\begin{frame}{TOPF}
  \begin{itemize}
    \item[] Deckel\footnote{Unsere Probleme löst die Zeit}:
      \begin{itemize}
      \item \emph{Jan Luca Naumann} (HU Berlin)
      \item Klemens Schmitt (TU Kaiserslautern)
      \end{itemize}
    \item[] Es gibt auch Henkel
    \end{itemize}
\end{frame}

\begin{frame}{TOPF}
  \begin{itemize}
  \item Der neue Server wurde eingerichtet, welche der ZaPF e.V. finanziert hat.
  \item Einen ausführlichen Bericht über die Tätigkeiten des TOPFs wird im eigenen AK vorgestellt.
  \end{itemize}
\end{frame}

\section{KommGrem}

\begin{frame}{KommGrem}
  \begin{itemize}
  \item[] ZaPF:
    \begin{itemize}
    \item \emph{Frederica Särdquist} (HU Berlin)
    \item Sonja Gehring (Uni Bonn)
    \end{itemize}
  \item[] jDPG:
    \begin{itemize}
    \item Eric Abraham (Jena)
    \item Merten Dahlkemper (Göttingen)
    \end{itemize}
  \end{itemize}
  \vspace{0.5cm}
%  \textbf{Sprecher}: 
\end{frame}

\section{LEUTE zur SACHE}

\begin{frame}{LEUTE zur SACHE\footnote{\textbf{L}ieblings \textbf{E}ngagierte in \textbf{U}ngewählter \textbf{T}askforc\textbf{E} zur \textbf{S}ach\textbf{A}rbeit am \textbf{CHE}}}
  \begin{itemize}
  \item \emph{Thomas Rudzki, Valentin Wohlfarth, Margret Heinze, Christian Hoffmann, Tim Luis Borck}
  \item 2 AKe vorbereitet: %wird noch aktualisiert
    \begin{itemize}
    \item CHE Info-Workshop
    \item CHE AK
    \end{itemize}
  \end{itemize}
\end{frame}

\section{ZaPF e.V.}

\begin{frame}{ZaPF e.V.}
  \begin{itemize}
  \item[] aktueller Vorstand:
    \begin{itemize}
    \item Frederike Kubandt (Uni Frankfurt) (Vorsitzende)
    \item Laura Lauf (Uni Frankfurt) (2. Vorsitzende)
    \item Patrick Haiber (Uni Konstanz) (Kassenwart)
    \item Jens Borgemeister (Uni Siegen)
    \item Jan Gräfje (Uni Heidelberg)
    \item Jan Luca Naumann (FU Berlin)
    \item Andreas Drotloff (Uni Würzburg)
    \item Tobias Löffler (Uni Düsseldorf)
    \item Lisa Dietrich (Uni Erlangen-Nürnberg)
    \end{itemize}
  \end{itemize}
\end{frame} 

\begin{frame}{ZaPF e.V.}
  \begin{itemize}
    \item Finanzschwache Fachschaften können nun einen Antrag an den ZaPF e.V. stellen
      \begin{itemize}
      \item Es können Teilnahmegebühren und Fahrtkosten gefördert werden.
      \end{itemize}
    \item Fachschaften und Personen können nun offiziell Fördermitglieder werden.
  \end{itemize}
\end{frame}

\end{document}
%%% Local Variables:
%%% mode: latex
%%% TeX-master: t
%%% End:
